\documentclass{article}
\usepackage{fontspec}
\usepackage{polyglossia}
\usepackage{gmverse}
\usepackage[blue]{showhyphens}
\setmainlanguage{basicfinnish}
\lefthyphenmin=1\righthyphenmin=1
\setmainfont{TeX Gyre Pagella}

\begin{document}
\begin{verse}
Mieleni minun tekevi, aivoni ajattelevi
lähteäni laulamahan, saa’ani sanelemahan,
sukuvirttä suoltamahan, lajivirttä laulamahan.
Sanat suussani sulavat, puhe’et putoelevat,
kielelleni kerkiävät, hampahilleni hajoovat.

Veli kulta, veikkoseni, kaunis kasvinkumppalini!
Lähe nyt kanssa laulamahan, saa kera sanelemahan
yhtehen yhyttyämme, kahta’alta käytyämme!
Harvoin yhtehen yhymme, saamme toinen toisihimme
näillä raukoilla rajoilla, poloisilla Pohjan mailla.

Lyökämme käsi kätehen, sormet sormien lomahan,
lauloaksemme hyviä, parahia pannaksemme,
kuulla noien kultaisien, tietä mielitehtoisien,
nuorisossa nousevassa, kansassa kasuavassa:
noita saamia sanoja, virsiä virittämiä
vyöltä vanhan Väinämöisen, alta ahjon Ilmarisen,
päästä kalvan Kaukomielen, Joukahaisen jousen tiestä,
Pohjan peltojen periltä, Kalevalan kankahilta.

Niit’ ennen isoni lauloi kirvesvartta vuollessansa;
niitä äitini opetti väätessänsä värttinätä,
minun lasna lattialla eessä polven pyöriessä,
maitopartana pahaisna, piimäsuuna pikkaraisna.
Sampo ei puuttunut sanoja eikä Louhi luottehia:
vanheni sanoihin sampo, katoi Louhi luottehisin,
virsihin Vipunen kuoli, Lemminkäinen leikkilöihin.

Viel’ on muitaki sanoja, ongelmoita oppimia:
tieohesta tempomia, kanervoista katkomia,
risukoista riipomia, vesoista vetelemiä,
päästä heinän hieromia, raitiolta ratkomia,
paimenessa käyessäni, lasna karjanlaitumilla,
metisillä mättähillä, kultaisilla kunnahilla,
mustan Muurikin jälessä, Kimmon kirjavan keralla.

Vilu mulle virttä virkkoi, sae saatteli runoja.
Virttä toista tuulet toivat, meren aaltoset ajoivat.
Linnut liitteli sanoja, puien latvat lausehia.

Ne minä kerälle käärin, sovittelin sommelolle.
Kerän pistin kelkkahani, sommelon rekoseheni;
ve’in kelkalla kotihin, rekosella riihen luoksi;
panin aitan parven päähän vaskisehen vakkasehen.

Viikon on virteni vilussa, kauan kaihossa sijaisnut.
Veänkö vilusta virret, lapan laulut pakkasesta,
tuon tupahan vakkaseni, rasian rahin nenähän,
alle kuulun kurkihirren, alle kaunihin katoksen,
aukaisen sanaisen arkun, virsilippahan viritän,
kerittelen pään kerältä, suorin solmun sommelolta?

Niin laulan hyvänki virren, kaunihinki kalkuttelen
ruoalta rukihiselta, oluelta ohraiselta.
Kun ei tuotane olutta, tarittane taarivettä,
laulan suulta laihemmalta, vetoselta vierettelen
tämän iltamme iloksi, päivän kuulun kunniaksi,
vaiko huomenen huviksi, uuen aamun alkeheksi.
\end{verse}
\end{document}
